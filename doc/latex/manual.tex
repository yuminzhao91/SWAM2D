\documentclass{smanual}
\usepackage{natbib}
\usepackage{graphics}
\usepackage{chngcntr}
\usepackage{sectsty}
\usepackage{rotating}
\usepackage{emptypage}

\let\leqslant=\leq


\newcommand{\psm}{small-scale physical modeling method}
\newcommand{\bialt}{\textit{BiAlt} }
\newcommand{\cmnt}[1]{}


\usepackage[dvipsnames]{xcolor}
\usepackage{minibox}

\usepackage{graphicx}
\usepackage{color}

\definecolor{vibrisblue1}{RGB}{90,151,190}
\definecolor{vibrisblue2}{RGB}{25,96,177}

\newcommand{\subtitle}{FDPSV}


\usepackage{makeidx}
\makeindex

\mdseries

\begin{document}

\begin{titlepage}
  \newpage
  \null
  \vskip 10.0em
  \let\center\flushleft
  {\noindent \huge{\bf FDPSV} \par}
  \vskip -0.5em
  {\noindent \rule{\textwidth}{0.3em} \par}

  {\hfill Documentation for FDPSV version 0.1 \par}
  
  {\hfill 11 January 2017 \par}
  
  \vfill
  
  {\noindent \Large{Damien Pageot}}
  \vskip -0.5em
  {\noindent \rule{\textwidth}{0.2em} \par}
  \vskip 1.0cm
\end{titlepage}

\tableofcontents
\newpage

\thispagestyle{empty}
\clearpage\newpage

% ----------------------------------------------------------------------
% INTRODUCTION
% ----------------------------------------------------------------------
\section{Introduction}
\index{Introduction}

% ----------------------------------------------------------------------
% THEORETICAL BACKGROUND
% ----------------------------------------------------------------------
\section{Theoretical background}

% ----------------------------------------------------------------------
% GETTING STARTED
% ----------------------------------------------------------------------
\section{Getting started}

% ----------------------------------------------------------------------
% NUMERICAL EXAMPLES
% ----------------------------------------------------------------------
\section{Numerical examples}

% ----------------------------------------------------------------------
% References
% ----------------------------------------------------------------------

\cite{aki2002quantitative}
\bibliography{references}

\printindex

\end{document}
